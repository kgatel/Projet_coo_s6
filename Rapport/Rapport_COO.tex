\documentclass[12,french]{report}
\usepackage{geometry}
\geometry{vmargin=3cm, hmargin=3cm}
\usepackage[T1]{fontenc}
\usepackage[utf8]{inputenc}
\usepackage[french]{babel}
\usepackage{graphicx}
\usepackage{amsmath}
\usepackage{amssymb}
\usepackage{sectsty}
\usepackage{authblk}
\usepackage{algpseudocode}
\usepackage{algorithm}
\usepackage{xspace}
\usepackage{mathtools}
\usepackage{mathrsfs}
\usepackage{enumitem}
\usepackage{titlesec}
\usepackage{hyperref}
\usepackage{xcolor}
\usepackage[justification=centering]{caption}
\usepackage{float}
\usepackage{tabto}

\usepackage{listings}
\usepackage{cleveref}

\renewcommand{\lstlistingname}{Code}
%\renewcommand{\figurename}{Fig.}

\lstdefinestyle{chstyle}{%
backgroundcolor=\color{gray!12},
basicstyle=\ttfamily\small,
showstringspaces=false,
numbers=left}

%\AddThinSpaceBeforeFootnotes
%\FrenchFootnotes

\titleformat{\chapter}[hang]{\bf\Huge}{\thechapter.}{2pc}{}
\titlespacing*{\chapter}{10pt}{0pt}{40pt}[0pt]
\newcommand{\HRule}{\rule{\linewidth}{0.5mm}}

\providecommand{\keywords}[1]{\textbf{\textit{Keywords:}} #1}
\bibliographystyle{apalike}

\usepackage{hyperref}

\begin{document}
\hypersetup{pdfborder=0 0 0}

\begin{titlepage}

\begin{center}
	\vspace*{\stretch{1}}
	\textsc{{\LARGE Institut national des sciences appliquées de Rouen} \\ 			\vspace{6mm} {\Large INSA de Rouen}} \\
	\vspace{5mm}
	\includegraphics[width=0.4\textwidth]{./Images/insa}\\[1.0 cm]

	\textsc{\Large Mini-projet COO - GM3}\\[0.6cm]

	% Title
	\HRule \\[0.5cm]
	{ \Huge \bfseries Réalisation d'un jeu de dames en Java}\\[0.2cm]
	\HRule \\[0.75cm]

	\includegraphics[width=0.7\textwidth]{./Images/Page_de_garde}\\[0.9 cm]

	% Author and supervisor
	\begin{minipage}{0.4\textwidth}
		\begin{flushleft} \large
			\emph{Auteurs:}\\
			Thibaut \textsc{André-Gallis} \\
			{\small\href{mailto:thibaut.andregallis@insa-rouen.fr}{thibaut.andregallis@insa-rouen.fr}} \\
			Kévin \textsc{Gatel} \\
			{\small\href{mailto:kevin.gatel@insa-rouen.fr}{kevin.gatel@insa-				rouen.fr}}
		\end{flushleft}
	\end{minipage}
	\begin{minipage}{0.4\textwidth}
		\begin{flushright} \large
			\emph{Enseignants:} \\
			Mathieu \textsc{Bourgais} \\
			{\small\href{mailto:mathieu.bourgais@insa-rouen.fr}								{mathieu.bourgais@insa-rouen.fr}}\\
			Habib \textsc{Abdulrab} \\
			{\small\href{mailto:habib.abdulrab@insa-rouen.fr}{habib.abdulrab@insa-rouen.fr}}
		\end{flushright}
	\end{minipage}
	\vspace*{\stretch{1}}

	\vfill
	{\large 16 Mai 2021}
\end{center}
\end{titlepage}

\tableofcontents

%\listoffigures

\renewcommand{\chaptername}{}
\chapter*{Introduction}

Le mini-projet que nous avons choisi est la réalisation du jeu de dames en Java.\\

Le jeu de dames est un célèbre jeu de réflexion joué à deux inventé pour la première fois en Egypte antique vers -1500 avant J-C.
Pour la version classique, le jeu de dames est composé d'un damier de 10 cases sur 10 cases de deux couleurs alternées et d'une vingtaine de pions par joueur disposés les uns en face des autres. Il existe également des variantes de ce jeu qui utilisent des damiers de 64 cases (8 sur 8) et 144 cases (12 par 12). 
Le but du jeu est alors de capturer ou d'immobiliser toutes les pièces du joueur adverse. \\

Le but de ce projet est d'utiliser habilement les concepts de la programmation objet, c'est-à-dire savoir écrire les différents diagrammes, maîtriser l'héritage entre les classes, le polymorphisme, la gestion des interfaces ou des classes arbitraires, etc...\\

Pour bien répondre au problème, nous allons d'abord regarder le \textbf{diagramme de cas} (Use-cases) et le \textbf{diagramme de classe} (UML) afin de comprendre l'organisation du code.\\

Ensuite, nous allons présenter et décrire le rôle de \textbf{chacune des classes} qui ont été nécessaires à la réalisation du projet.\\

Enfin, nous verrons \textbf{les différents problèmes} que nous avons pu rencontrer lors de la programmation du jeu.


\chapter{Les diagrammes}

\section{Use-cases}

\begin{figure}[H]
	\center
	\includegraphics[width=0.7\textwidth]{./Images/Use-cases}
	\caption{Diagramme de cas pour le jeu de dames}
\end{figure}\vspace{0.2cm}

Les joueurs commencent par choisir la variante de leur choix pour jouer au jeu de dames. Ils ont le choix entre :

\begin{itemize}[label=\textbullet]
	\item Une partie rapide : damier 8 sur 8 où chaque joueur possède 16 pièces.
	\item Une partie classique : damier 10 sur 10 où chaque joueur possède 20 pièces.
	\item Une partie longue : damier 12 sur 12 où chaque joueur possède 24 pièces.
\end{itemize}\vspace{0.3cm}

Ils choisissent ensuite s'ils veulent ou non obliger le saut de pièce (dans la version classique le saut est obligatoire).\\

La partie se lance et les joueurs déplacent chacun leur tour une pièce de leur couleur jusqu'à ce que la partie se termine.


\section{UML}

\begin{figure}[H]
	\center
	\includegraphics[width=1\textwidth]{./Images/UML}
	\caption{Diagramme de classe pour le jeu de dames}
\end{figure}\vspace{0.2cm}

La classe \textit{Menu.java} est la première classe utilisée lors du lancement du programme. On définit ici la taille du plateau, on demande aux joueurs s'ils veulent ou non obliger les sauts et on récupère leur pseudo respectif.\\

La classe \textit{Damier.java} est la classe centrale du projet. Elle permet le déplacement des pièces, l'affichage du damier et la gestion des cases. Grâce à la classe \textit{JPanel}, elle hérite de la méthode prédéfinie \textit{void paint(Graphics g)} qu'on a modifié en utilisant les méthodes \textit{void dessinerPiece()} et \textit{void dessinerCase()} et permet ainsi l'affichage du plateau.\\

La méthode \textit{void afficherDeplacement(x,y : int)} va colorier les cases disponibles d'une certaine couleur pour le déplacement d'une pièce de coordonnées (x,y). On coloriera en bleu un déplacement sans saut de pièce et en rose pour un déplacement avec saut de pièce.\\



La méthode \textit{void déplacer(x,y : int)} va permettre ou non le déplacement d'un pion déjà sélectionné sur une case de coordonnées (x,y). Le déplacement sera autorisé si la case a été colorée en bleu ou en rose. Si la case (x,y) est rose, on utilise la méthode \textit{boolean sautPossible} pour savoir s'il peut continuer à jouer, à savoir s'il peut sauter des pièces, le tour du joueur sera alors inchangé.

A chaque saut, on utilise la méthode \textit{Coordonnees pieceMangee(x,y,i,j : int)} qui retourne les coordonnées de la pièce mangée, pour ensuite l'enlever du plateau.\\

A la fin du tour, on utilise la méthode \textit{boolean peutEtreMange(x,y : int)} sur chaque pièce du joueur. S'il s'avère qu'une pièce peut être mangé au tour d'après, si l'option \textit{sautObligatoire} a été choisie par les joueurs en début de partie, tout déplacement du joueur suivant n'étant pas un saut est bloqué, un message d'erreur est affiché si nécessaire.\\

La méthode \textit{boolean partieFinie()} retourne vrai si un joueur ne possède plus de pion ou ne peut plus en déplacer. La partie se termine et un message affiche le vainqueur du jeu.\\

La classe \textit{Souris.java} va permettre aux joueurs de déplacer leur pièce grâce au clique de la souris. En effet, elle est issue de l'interface \textit{MouseListener} qui comprend de nombreuses méthodes prédéfinies. En modifiant la méthode \textit{void mousePressed(MouseEvent arg0)} on est capable de récupérer l'endroit où le joueur clique pour ensuite effectuer des actions sur le damier placé en attribut.


\chapter{Les classes}

\section{Case}

\section{Coordonnes}

\section{Couleur}

\section{Damier}

\section{Joueur}

\section{Lanceur}

\section{Menu}

\section{Piece}

\section{Pion}

\section{Reine}

\section{Souris}



\chapter{Problèmes rencontrés}

\chapter*{Conclusion}
\addcontentsline{toc}{chapter}{Conclusion}

\chapter*{Annexe}

\end{document}
